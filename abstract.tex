% Word limit: 250
The distribution of rotation periods of K and M stars, measured from light
curves obtained from the \kepler\ spacecraft, has a sharp mass-dependent gap
at around 10-20 days.
This gap traces a line of constant age and constant Rossby number in the
rotation period-effective temperature plane and its discovery has disrupted
our understanding of stellar evolution.
The origin of this gap is unknown, but possible physical explanations include
a discontinuity in the local star formation history, or a discontinuity in the
magnetic braking evolution of stars.
An alternative explanation for the rotation period gap is measurement error
caused by confounders such as binary companions or aliasing.
For example, the lower rotation sequence could be a reflection of the upper
sequence, caused by incorrect measurements at half the true period.
In this paper, we rule out the possibility that this gap could be caused by
incorrect period measurements or binary companions, by showing that the
rapidly rotating stars are kinematically young and velocity dispersion
increases smoothly with rotation period and gyrochronal age.
